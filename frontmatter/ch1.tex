%chapter 1
\chapter{Transportation System}
%
\section{Introduction}
Transportation is a crucial aspect of a nation's development and prosperity. However, with the advent of modern means of transportation, managing its operation has become more complex, so there is ample opportunity for engineers to devise new operational and management techniques for the smooth control of various modes of transportation of a country. 

\section{History of Transportation}
\subsection*{Early Boats}
\par
The first mode of transportation was created in the effort to traverse water: boats. Those who colonized Australia roughly 60,000–40,000 years ago have been credited as the first people to cross the sea, though there is some evidence that seafaring trips were carried out as far back as 900,000 years ago.
\\\\
\par
The earliest known boats were simple logboats, also referred to as dugouts, which were made by hollowing out a tree trunk. Evidence for these floating vehicles comes from artifacts that date back to around 10,000–7,000 years ago. The Pesse canoe—a logboat—is the oldest boat unearthed and dates as far back as 7600 BCE. Rafts have been around nearly as long, with artifacts showing them in use for at least 8,000 years.
\\\\
\subsection*{Horses and Wheeled Vehicles}
\par
Next, came horses. While it’s difficult to pinpoint exactly when humans first began domesticating them as a means of getting around and transporting goods, experts generally go by the emergence of certain human biological and cultural markers that indicate when such practices started to take place.
\\\\
\par
Based on changes in teeth records, butchering activities, shifts in settlement patterns, and historic depictions, experts believe that domestication took place around 4000 BCE. Genetic evidence from horses, including changes in musculature and cognitive function, support this.
\\\\
\par
It was also roughly around this period that the wheel was invented. Archaeological records show that the first wheeled vehicles were in use around 3500 BCE, with evidence of the existence of such contraptions found in Mesopotamia, the Northern Caucuses, and Central Europe. The earliest well-dated artifact from that time period is the "Bronocice pot," a ceramic vase that depicts a four-wheeled wagon that featured two axles. It was unearthed in southern Poland.
\\\\
\subsection*{Steam Engines}
\par
In 1769, the Watt steam engine changed everything. Boats were among the first to take advantage of steam-generated power; in 1783, a French inventor by the name of Claude de Jouffroy built the "Pyroscaphe," the world’s first steamship. But despite successfully making trips up and down the river and carrying passengers as part of a demonstration, there wasn’t enough interest to fund further development.
\\\\
\par
While other inventors tried to make steamships that were practical enough for mass transport, it was American Robert Fulton who furthered the technology to where it was commercially viable. In 1807, the Clermont completed a 150-mile trip from New York City to Albany that took 32 hours, with the average speed clocking in at about five miles per hour. Within a few years, Fulton and company would offer regular passenger and freight service between New Orleans, Louisiana, and Natchez, Mississippi.
\\\\
\par
Back in 1769, another Frenchman named Nicolas Joseph Cugnot attempted to adapt steam engine technology to a road vehicle—the result was the invention of the first automobile. However, the heavy engine added so much weight to the vehicle that it wasn't practical. It had a top speed of 2.5 miles per hour.
\\\\
\par
Another effort to repurpose the steam engine for a different means of personal transport resulted in the "Roper Steam Velocipede." Developed in 1867, the two-wheeled steam-powered bicycle is considered by many historians to be the world’s first motorcycle.
\\\\
\subsection*{Locomotives}
\par
One mode of land transport powered by a steam engine that did go mainstream was the locomotive. In 1801, British inventor Richard Trevithick unveiled the world’s first road locomotive—called the “Puffing Devil”—and used it to give six passengers a ride to a nearby village. It was three years later that Trevithick first demonstrated a locomotive that ran on rails, and another one that hauled 10 tons of iron to the community of Penydarren, Wales, to a small village called Abercynon.
\\\\
\par
It took a fellow Brit—a civil and mechanical engineer named George Stephenson—to turn locomotives into a form of mass transport. In 1812, Matthew Murray of Holbeck designed and built the first commercially successful steam locomotive, “The Salamanca,” and Stephenson wanted to take the technology a step further. So in 1814, Stephenson designed the "Blücher," an eight-wagon locomotive capable of hauling 30 tons of coal uphill at a speed of four miles per hour.
\\\\
\par
By 1824, Stephenson improved the efficiency of his locomotive designs to where he was commissioned by the Stockton and Darlington Railway to build the first steam locomotive to carry passengers on a public rail line, the aptly named "Locomotion No. 1." Six years later, he opened the Liverpool and Manchester Railway, the first public inter-city railway line serviced by steam locomotives. His notable accomplishments also include establishing the standard for rail spacing for most of the railways in use today. No wonder he’s been hailed as "Father of Railways."
\\\\
\subsection*{Submarines}
\par
Technically speaking, the first navigable submarine was invented in 1620 by Dutchman Cornelis Drebbel. Built for the English Royal Navy, Drebbel’s submarine could stay submerged for up to three hours and was propelled by oars. However, the submarine was never used in combat, and it wasn’t until the turn of the 20th century that designs leading to practical and widely used submersible vehicles were realized.
\\\\
\par
Along the way, there were important milestones such as the launch of the hand-powered, egg-shaped "Turtle" in 1776, the first military submarine used in combat. There was also the French Navy submarine "Plongeur," the first mechanically powered submarine.
\\\\
\par
Finally, in 1888, the Spanish Navy launched the "Peral," the first electric, battery-powered submarine, which also so happened to be the first fully capable military submarine. Built by a Spanish engineer and sailor named Isaac Peral, it was equipped with a torpedo tube, two torpedoes, an air regeneration system, and the first fully reliable underwater navigation system, and it posted an underwater speed of 3.5 miles per hour.
\\\\
\subsection*{Aircraft}
\par
The start of the twentieth century was truly the dawn of a new era in the history of transportation as two American brothers, Orville and Wilbur Wright, pulled off the first official powered flight in 1903. In essence, they invented the world’s first airplane. Transport via aircraft took off from there with airplanes being put into service within a few short years during World War I. In 1919, British aviators John Alcock and Arthur Brown completed the first transatlantic flight, crossing from Canada to Ireland. The same year, passengers were able to fly internationally for the first time.
\\\\
\par
Around the same time that the Wright brothers were taking flight, French inventor Paul Cornu started developing a rotorcraft. And on November 13, 1907, his "Cornu" helicopter, made of little more than some tubing, an engine, and rotary wings, achieved a lift height of about one foot while staying airborne for about 20 seconds. With that, Cornu would lay claim to having piloted the first helicopter flight.