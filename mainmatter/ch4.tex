%chapter 4
\chapter{Insurance}
\section{Introduction}
There is no single definition of insurance. Insurance can be defined from the viewpoint of several disciplines, including law, economics, history, actuarial science, risk theory, and sociology. But each possible definition will not be examined at this point. Instead, we will examine the common elements that are typically present in any insurance plan. However, before proceeding, a working definition of insurance—one that captures the essential characteristics of a true insurance plan—must be established.\\
\par
After careful study, the Commission on Insurance
Terminology of the American Risk and Insurance Association has defined insurance as follows. \textit{\textbf{Insurance is the pooling of fortuitous losses by transfer of such risks to insurers, who agree to indemnify insureds for such losses, to provide other pecuniary benefits on their occurrence, or to render services connected with the risk}}. Although this lengthy definition may not be acceptable to all risk managers and insurance scholars, it is useful for analyzing the common elements of a true insurance plan.
%
\section{Basic Characteristics of Insurance}
Based on the preceding definition, an insurance plan or arrangement typically includes the following
characteristics:
\subsection{Pooling of Losses}
Pooling or the sharing of losses is the essence of insurance. \textit{Pooling is the spreading of losses incurred by the few over the entire group, so that in the process, average loss is substituted for actual loss.} In addition, pooling involves the grouping of a large number of exposure units so that the law of large numbers can operate to provide a substantially accurate prediction of future losses. Ideally, there should be a large number of similar, but not necessarily identical, exposure units that are subject to the same perils. Thus, pooling implies (1) the sharing of losses by the entire group and (2) the prediction of future losses with some accuracy based on the law of large numbers.\\
\par
The primary purpose of pooling, or the sharing of losses, is to reduce the variation in possible outcomes as measured by the standard deviation or some other measure of dispersion, which reduces risk. For example, assume that two freight business owners each own an identical storage building valued at \$50,000. Assume there is a 10\% chance in any year that each building will be destroyed by a peril, and that a loss to either building is an independent event. The expected annual loss for each owner is \$5,000 as shown below:\\
Expected Loss = $ 0.90 \times \$0 + 0.10 \times \$50000 = \$5000 $\\
\par
A common measure of risk is the standard deviation, which is the square root of the variance. The standard deviation (SD) for the expected value of the loss is \$15,000, as shown below:\\\\
SD = $\sqrt{ 0.9 \times (0 - 5000)^2 + 0.1 \times (50000 - 5000)^2} = \$15000$\\
\par
Suppose instead of bearing the risk of loss individually, the two owners decide to pool (combine) their loss exposures, and each agrees to pay an equal share of any loss that might occur. Under this scenario, there are four possible outcomes:
\begin{center}
	\begin{tabular}{c c}
		Possible Outcomes & Probability\\
		\hline
		Neither Building is destroyed & $0.9 \times 0.9 = 0.81$\\
		First building is destroyed and second building is intact & $0.1 \times 0.9 = 0.09$\\
		First building is intact and second building is destroyed & $0.9 \times 0.1 = 0.09$\\
		Both buildings are destroyed & $ 0.1 \times 0.1 = 0.01$
	\end{tabular}
\end{center}
If neither building is destroyed, the loss for each owner is \$0. If one building is destroyed, each owner pays \$25,000. If both buildings are destroyed, each owner must pay \$50,000. The expected loss for each owner remains \$5,000 as shown below:\\\\
Expected loss = $ 0.81 \ltimes \$0 + 0.09 \ltimes \$25000 + 0.09 \ltimes \$25000 + 0.01 \times \$50000 = \$5000$\\
\par
Note that while the expected loss remains the same, the probability of the extreme values, \$0 and \$50,000, have declined. The reduced probability of the extreme values is reflected in a lower standard deviation as shown below:\\\\
SD = $ \sqrt{0.81(0 - 5000) + 0.09(25000 - 5000) + 0.09(25000 - 5000) + 0.01(50000 -5000)}$\\
SD = \$10607\\
\par
Thus, as additional individuals are added to the pooling arrangement, the standard deviation continues to decline while the expected value of the loss remains unchanged. For example, with a pool of 100 insureds, the standard deviation is \$1,500; with a pool of 1,000 insureds, the standard deviation is \$474; and with a pool of 10,000, the standard deviation is \$150.\\
\par
In addition, by pooling or combining the loss experience of a large number of exposure units, an insurer may be able to predict future losses with greater accuracy. From the viewpoint of the insurer, if future losses can be predicted, objective risk is reduced. Thus, another characteristic often found in many lines of insurance is risk reduction based on the law of large numbers.\\
\par
\textit{The law of large numbers states that the greater the number of exposures, the more closely will the actual results approach the probable results that are expected from an infinite number of exposures}. For example, if you flip a balanced coin into the air, the a priori probability of getting “heads” is 0.5. If you flip the coin only 10 times, you may get heads eight times. Although the observed probability of getting heads is 0.8, the true probability is still 0.5. If the coin were flipped 1 million times, however, the actual number of heads would be approximately 500,000. Thus, as the number of random tosses increases, the actual results approach the expected results.
%
\subsection{Payment of Fortuitous Losses}
A second characteristic of private insurance is the payment of fortuitous losses. Most insurance policies exclude intentional losses. \textit{A fortuitous loss is one that is unforeseen and unexpected by the insured and occurs as a result of chance.} In other words, the loss must be accidental. The law of large numbers is based on the assumption that losses are accidental and occur randomly. For example, a person may slip on an icy sidewalk and break a leg. The loss would be fortuitous.
%
\subsection{Risk Transfer}
Risk transfer is another essential element of insurance. With the exception of self-insurance, a true insurance plan always involves risk transfer. \textit{Risk transfer means that a pure risk is transferred from the insured to the insurer, who typically is in a stronger financial position to pay the loss than the insured.} From the viewpoint of the individual, pure risks that are typically transferred to insurers include the risk of premature death, excessive longevity, poor health, disability, destruction and theft of property, and personal liability lawsuits.
%
\subsection{Indemnification}
A final characteristic of insurance is indemnification for losses. \textit{Indemnification means that the insured is restored to his or her approximate financial position prior to the occurrence of the loss.} Thus, if your home burns in a fire, a homeowners policy will indemnify you or restore you to your previous position. If you are sued because of the negligent operation of an automobile, your auto liability insurance policy will pay those sums that you are legally obligated to pay. Similarly, if you become seriously disabled, a disability-income insurance policy will restore at least part of the lost wages.
%
\section{Characteristics of an Ideally Insurable Risk}
Private insurers generally insure only pure risks. However, some pure risks are not privately insurable. From the viewpoint of a private insurer, an insurable risk ideally should have certain characteristics. There are ideally six characteristics of an insurable risk:
%
\subsection{Large Number of Exposure Units}
The first requirement of an insurable risk is a large number of exposure units. Ideally, there should be a large group of roughly similar, but not necessarily identical, exposure units that are subject to the same peril or group of perils. For example, a large number of wood frame dwellings in a city can be grouped together for purposes of providing property insurance on the dwellings.\\
\par
The purpose of this first requirement is to enable the insurer to predict losses based on the law of large numbers. Loss data can be compiled over time, and losses for the group as a whole can be predicted with some accuracy. The loss costs can then be spread over all insureds in the underwriting class.
%
\subsection{Accidental and Unintentional Loss}
A second requirement is that the loss should be accidental and unintentional; ideally, the loss should be unforeseen and unexpected by the insured and outside of the insured’s control. Thus, if an individual deliberately causes a loss, he or she should not be indemnified for the loss.\\
\par
There are several reasons for this requirement. First, the loss should be accidental because the law of large numbers is based on the random occurrence of events. A deliberately caused loss is not a random event because the insured knows when the loss will occur. Thus, prediction of future experience may be highly inaccurate if a large number of intentional or nonrandom losses occur. Second, moral hazard is increased if the insured deliberately intends to cause a loss. Finally, it is poor public policy to allow insureds to collect for intentional losses.
%
\subsection{Determinable and Measurable Loss}
A third requirement is that the loss should be both determinable and measurable. This means the loss should be definite as to cause, time, place, and amount. Life insurance, in most cases, meets this requirement easily. The cause and time of death can usually be readily determined, and if the person is insured, the face amount of the life insurance policy is the amount paid.\\
\par
Some losses, however, are difficult to determine and measure. For example, under a disability-income policy, the insurer promises to pay a monthly benefit to the disabled person if the definition of disability stated in the policy is satisfied. Some dishonest claimants may deliberately fake sickness or injury to collect from the insurer. Even if the claim is legitimate, the insurer must still determine whether the insured satisfies the definition of disability stated in the policy. Sickness and disability are highly subjective, and the same event can affect two persons quite differently. For example, two accountants who are insured under separate disability-income contracts may be injured in an auto accident, and both may be classified as totally disabled. One accountant, however, may be more determined to return to work. If that accountant undergoes rehabilitation and returns to work, the disability-income benefits will terminate. Meanwhile, the other accountant would still continue to receive disability-income benefits according to the terms of the policy. In short, it is often difficult to determine when a person is actually disabled. However, all losses ideally should be both determinable and measurable.\\
\par
The basic purpose of this requirement is to enable an insurer to determine if the loss is covered under the policy, and if it is covered, how much should be paid. For example, assume that Shannon has an expensive fur coat that is insured under a homeowners policy. It makes a great deal of difference to the insurer if a thief breaks into her home and steals the coat, or the coat is missing because her husband stored it in a drycleaning establishment but forgot to tell her. The loss is covered in the first example but not in the second.
%
\subsection{No Catastrophic Loss}
The fourth requirement is that ideally the loss should not be catastrophic. This means that a large proportion of exposure units should not incur losses at the same time. As we stated earlier, pooling is the essence of insurance. If most or all of the exposure units in a certain class simultaneously incur a loss, then the pooling technique breaks down and becomes unworkable. Premiums must be increased to prohibitive levels, and the insurance technique is no longer a viable arrangement by which losses of the few are spread over the entire group.\\
\par
Insurers ideally wish to avoid all catastrophic losses. In reality, however, that is impossible, because catastrophic losses periodically result from floods, hurricanes, tornadoes, earthquakes, forest fires, and other natural disasters. Catastrophic losses can also result from acts of terrorism.\\
\par
Several approaches are available for meeting the problem of a catastrophic loss. First, reinsurance can be used by which insurance companies are indemnified by reinsurers for catastrophic losses. Reinsurance is an arrangement by which the primary insurer that initially writes the insurance transfers to another insurer (called the reinsurer) part or all of the potential losses associated with such insurance. The reinsurer is then responsible for the payment of its share of the loss.\\
\par
Second, insurers can avoid the concentration of
risk by dispersing their coverage over a large geographical area. The concentration of loss exposures in a geographical area exposed to frequent floods, earthquakes, hurricanes, or other natural disasters can result in periodic catastrophic losses. If the loss exposures are geographically dispersed, the possibility of a catastrophic loss is reduced.\\
\par
Finally, financial instruments are now available for dealing with catastrophic losses. These instruments include catastrophe bonds, which are designed to help fund catastrophic losses.
%
\subsection{Calculable Chance of Loss}
Another requirement is that the chance of loss should be calculable. The insurer must be able to calculate
both the average frequency and the average severity of future losses with some accuracy. This requirement is necessary so that a proper premium can be charged that is sufficient to pay all claims and expenses and yields a profit during the policy period.\\
\par
Certain losses, however, are difficult to insure because the chance of loss cannot be accurately estimated, and the potential for a catastrophic loss is present. For example, floods, wars, and cyclical unemployment occur on an irregular basis, and prediction of the average frequency and severity of losses is difficult. Thus, without government assistance, these losses are difficult for private carriers to insure.
%
\subsection{Economically Feasible Premium}
A final requirement is that the premium should be economically feasible. The insured must be able to afford the premium. In addition, for the insurance to be an attractive purchase, the premiums paid must be substantially less than the face value, or amount, of the policy.\\
\par
To have an economically feasible premium, the chance of loss must be relatively low. One view is that if the chance of loss exceeds 40 percent, the cost of the policy will exceed the amount that the insurer must pay under the contract. For example, an insurer could issue a \$1,000 life insurance policy on a man who is age 99, but the pure premium would be close to that amount, and an additional amount for expenses would also have to be added. The total premium would exceed the face amount of insurance.\\
\par
Based on the preceding requirements, most personal risks, property risks, and liability risks can be privately insured because the ideal characteristics of an insurable risk generally can be met. In contrast, most market risks, financial risks, production risks, and political risks are difficult to insure by private insurers. These risks are speculative, and calculation of a correct premium may be difficult because the chance of loss cannot be accurately estimated. For instance, insurance that protects a retailer against loss because of a change in consumer tastes, such as a style change, generally is not available. Accurate loss data are not available. Thus, it would be difficult to calculate an accurate premium. The premium charged may or may not be adequate to pay all losses and expenses. Since private insurers are in business to make a profit, certain risks are difficult to insure because of the possibility of substantial losses.
%
\section{Types of Insurance}
Insurance can be classified as either private or government insurance. Private insurance includes life and health insurance as well as property and liability insurance. Government insurance includes social insurance programs and other government insurance plans.
%
\subsection{Private Insurance}
%
\subsubsection{Life Insurance}
Life insurance pays death benefits to designated
beneficiaries when the insured dies. The benefits pay for funeral expenses, uninsured medical bills, estate taxes, and other expenses. The death proceeds can also provide periodic income payments to the deceased’s beneficiary. Life insurers also sell annuities, individual retirement account (IRA) plans, 401(k) plans, and individual and group retirement plans. Some life insurers also sell (1) individual and group health insurance plans that cover medical expenses because of sickness or injury, (2) disability income plans that replace income lost during a period of disability, and (3) long-term care policies that cover care in nursing facilities.
%
\subsubsection{Health Insurance}
Although many life insurers we described also sell some type of individual or group health insurance plan, the health insurance industry overall is highly specialized and controlled by a relatively small number of insurers. These companies include Blue Cross Blue Shield Association, AETNA, United Health Group, and Well Point. Medical expense plans pay for hospital and surgical expenses, physician fees, prescription drugs, and a wide variety of additional medical costs.
%
\subsubsection{Property and Liability Insurance}
Property insurance indemnifies property owners against the loss or damage of real or personal property caused by various perils, such as fire, lightning, windstorm, or tornado. Liability insurance covers the insured’s legal liability arising out of property damage or bodily injury to others; legal defense costs are also paid.\\
\par
Property and liability insurance is also called property and casualty insurance. In practice, nonlife insurers typically use the term property and casualty insurance (rather than property and liability insurance) to describe the various coverages and operating results. Casualty insurance is a broad field of insurance that covers whatever is not covered by fire, marine, and life insurance; casualty lines include auto, liability, burglary and theft, workers compensation, and health insurance.
%
\subsection{Government Insurance}
Numerous government insurance programs are in operation at the present time. Social insurance is the one discussed in this chapter, whereas other government insurance also exist but differ as per the national and state policy of a country.
%
\subsubsection{Social Insurance}
Social insurance programs are government insurance programs with certain characteristics that distinguish them from other government insurance plans. These programs are financed entirely or in large part by mandatory contributions from employers, employees, or both, and not primarily by the general revenues of government. The contributions are usually earmarked for special trust funds; the benefits, in turn, are paid from these funds. In addition, the right to receive benefits is ordinarily derived from or linked to the recipient’s past contributions or coverage under the program; the benefits and contributions generally vary among the beneficiaries according to their prior earnings, but the benefits are heavily weighted in favor of low-income groups. Moreover, most social insurance programs are compulsory. Covered workers and employers are required by law to pay contributions and participate in the programs. Finally, eligibility requirements and benefit rights are usually prescribed exactly by statute, leaving little room for administrative discretion in the award of benefits.
\%
\section{Readings from Insurance Regulation, 2049 (1993)}
The following are the readings extracted from \textit{"Insurance Regulation, 2049"}.
\subsection{Definitions}
Unless the subject or context otherwise requires, in this Regulation :
\begin{itemize}
	\item  "Act" means the Insurance Act, 2049
	\item Certificate" means an Insurer Registration Certificate provided to an Insurer for
	operating the Insurance Business by registering him as an Insurer pursuant to Rule 8
	\item "Premium" means the Insurance Premium to be collected by the Insurer from the insured i consideration of the Insurance Business
	\item "Advisory Committee" means the Insurance Tariff Advisory Committee constituted
	pursuant to section 41 of the Act
\end{itemize}
%
\subsection{Categories of Insurance Business}
Subject to the provisions made in the Act and this Regulation, the Insurance Business to be operated by an Insurer shall be divided into the following categories :
\begin{itemize}
	\item Life Insurance Business
	\item Non-Life Insurance Business
	\item Re-Insurance Business
\end{itemize}
Notwithstanding anything contained in sub-rule (1), Nepal Government may prescribe other categories of Insurance Business as required on the advise of the Board.
%
\subsubsection{Life Insurance Business}
\begin{enumerate}
	\item The Insurer may operate the following Insurance Business under the Life Insurance Business:
	\begin{enumerate}
		\item Whole Life Insurance
		\item Endowment Life Insurance
		\item Term Life Insurance
	\end{enumerate}
	\item Notwithstanding anything contained in sub-rule (1), the Board may prescribe other
	categories of the Life Insurance Business as required.
	\item  The conditions and privileges of the Life Insurance Policy to be executed pursuant to this
	Rule shall be as specified by the Board.
\end{enumerate}
%
\subsubsection{Non-Life Insurance Business}
\begin{enumerate}
	\item The Insurer may operate the following Insurance Business under the Non-Life Insurance Business :
	\begin{enumerate}
		\item Fire Insurance
		\item Motor Insurance
		\item Marine Insurance
		\item Engineering and Contractor's Risk Insurance
		\item Aviation Insurance
		\item Miscellaneous Insurance
	\end{enumerate}
	\item Notwithstanding anything contained in sub-rule (1), the Board may prescribe other categories of Non-Life Insurance Business as required.
	\item The conditions and privileges of the Non-Life Insurance Policy to be executed pursuant to this rule shall be as specified by the Board.

\end{enumerate}
%
\subsubsection{Re-Insurance Business}
\begin{enumerate}
	\item The Insurer may re-insure the risks which are in excess from the risks assumed by it.
	\item The Categories of Re-insurance Business to be made pursuant to sub-rule (1) and other arrangement shall be as specified by the Board
\end{enumerate}
%
\subsection{Provisions Relating to a Surveyor}
\subsubsection{Application to be submitted for surveyor's license}
Any person or any corporate body having a qualification, as mentioned in regulation, and desirous to work as a Surveyor pursuant to sub-section (1) of Section \# 30A of the Act, shall submit an application to the office of the Board in the format of Schedule -12.
%
\subsubsection{Surveyor's License to be provided}
\begin{enumerate}
	\item After receiving an application for the Surveyor's license pursuant to rule-26, the Board shall make an inquiry whether the applicant is qualified or not pursuant to Rule 28, and if it deems appropriate to provide the Surveyor's license to him it shall register his name as a Surveyor in the registration-book pursuant to Schedule - 13.
	\item After making the registration of the name of the applicant on the registration-book pursuant
	to sub-rule (1), the Board shall provide a Surveyor's license to the applicant in the format of Schedule - 14 by collecting from the applicant a fee of twelve thousand Rupees for the class A license, that of nine thousand Rupees for the class B license, that of seven
	thousand rupees for the class C license and that of five thousand Rupees for the class D license.
\end{enumerate}
%
\subsubsection{Qualifications of a Surveyor}
\begin{enumerate}
	\item An applicant desirous of making an application for the Surveyor's License pursuant to Rule 26 shall have possessed any one of the following qualifications :
	\begin{enumerate}
		\item Having gained at least ten years of work experiences on the Insurance Business, holding
		an officer level post at the office of any Insurer
		\item  Having possessed at least a Bachelor Degree in Engineering subject
		\item Having possessed at least a Bachelor Degree in Insurance subject from a Chartered Insurance Institute of international standard or from an organization recognized by such
		institute, or
		\item Having passed the Chartered Accountancy Examination
	\end{enumerate}
	\item An applicant having possessed the qualification as referred to in clause (a) or (b) or (c) or
	(d) of sub-rule (1) shall, prior to obtaining the Surveyor's License, have obtained a certificate indicating his participation in, and completion of, the surveyor training conducted by the Board
\end{enumerate}
%
\subsubsection{Classification of the Surveyor}
The Surveyors who have been working as Surveyors after having obtained the Surveyor's license and completed the following period shall be classified as follows and provided with the Surveyor's License, after the commencement of this Regulation:
\begin{itemize}
	\item  A Surveyor who has regularly worked as a Surveyor for a period more than fifteen years, Class "A"
	\item A Surveyor who has regularly worked as a Surveyor for a period from ten to fifteen years, Class "B"
	\item A Surveyor who has regularly worked as a Surveyor for a period From five to ten years, Class "C"
	\item A Surveyor who has regularly worked as a Surveyor for a period of five years, Class "D"
\end{itemize}
%
\subsubsection{Provisions relating to the renewal of Surveyor's License}
\begin{enumerate}
	\item The Surveyor shall submit an application to the office of the Board in the format of Schedule-15 along with the renewal fee of →twelve thousand Rupees for "A" class license, nine thousand Rupees for "B" class license, seven thousand Rupees for "C" class license
	and five thousand Rupees for "D" class license within the time-limit pursuant to subsection (1) of Section 31 of the Act for the renewal of the License. On receipt of such application, the Board shall renew the Surveyor's license
	\item If a Surveyor has submitted an application to the office of the Board stating the reasons for
	inability to submit an application for the renewal of his license within the time-limit pursuant to sub-rule (1), and if the reasons are found to be appropriate, the Board may renew the license by receiving an additional fee of five hundred Rupees for first two months from the date of expiry of the renewal time-limit and after that fifty Rupees per day for up to four months
\end{enumerate}
%
\subsubsection{Limitation of Survey}
\begin{enumerate}
	\item The limitation relating to survey which a surveyor of each category pursuant to Rule 28A can make shall be as prescribed by the Board
	\item The Code of Conduct of the Surveyor shall be as prescribed by the Board
\end{enumerate}
%
\subsubsection{In the case of cancellation of Surveyor's License}
If the Surveyor's license is cancelled pursuant to Section 33 of the Act, no other Surveyor's license shall be provided to him to work as a Surveyor up to a period of five years from the date of such cancellation.
%%%
\subsection{Provisions relating to payment against Insurance Claim}
\subsubsection{Process against the payment of Life Insurance Claim}
\begin{enumerate}
	\item The Insurer shall issue a discharge voucher in the name of the Insured who has already paid the last installment of the Life Insurance Premium requesting him to come to collect payment against the claim along with the Insurance Policy and other documents required for making payment against such Life Insurance claim within fifteen days from the date of payment of such installment
	\item In case an Insured submits the Insurance Policy and other documents including the
	discharge voucher to the Insurer for the payment of claim against the Life Insurance claim pursuant to sub-rule (1), the Insurer shall conduct an inquiry as required and make a payment against the Life Insurance claim within seven days from the date of expiry of the
	period of the Life Insurance Policy
	\item In case any person who has taken up an Insurance Policy dies before the expiry of the
	period of the Insurance Policy, the person designated by him, if any, and in case no person has been designated, the nearest heir from among the persons mentioned in sub-section (1) of Section 38 of the Act shall submit an application for the payment against the claim to the
	Insurer to receive the amount of the Life Insurance stating the details as follows :
	\begin{enumerate}
		\item The details relating to the claim
		\item  A Certificate of death of the insured
		\item In case the insured has died in an accident and if such risk is covered by the Life Insurance, the postmortem report of the government physician relating to the cause of death, and if there is no such report, a report of the police
		\item A certificate of relationship with the insured
		\item The documents regarding the certification of the age in case the age has not been certified
		\item Other details specified by the Board
	\end{enumerate}
	\item After the receipt of the application pursuant to sub-rule (3), the Insurer shall make an inquiry into the details including the documents submitted regarding to the claim of Life Insurance, and shall examine other matters also if necessary, and shall determine the liability within fifteen days from the date of receipt of such documents by it and shall issue the discharge voucher in the name of the applicant requesting him to come to collect the payment against the claim. The Insurer shall make the payment against the Insurance
	claim within fifteen days from the date of receipt of the discharge voucher from the applicant
	\item If it is found, while making an inquiry into the details pursuant to sub-rule (4) that the
	Insurance claim need not to be paid by determining the liability, the Insurer shall provide
	a written information to the applicant clearly stating the reasons thereof.
\end{enumerate}
%
\subsubsection{Process of payment against Non-Life Insurance Claim}
\begin{enumerate}
	\item  If any claim has to be made under the Insurance Policy by an Insured who has taken up a Non-Life Insurance Policy, the Insured shall submit an application to the Insurer stating all the details relating to it
	\item On receipt of an application of the Insured for the payment against the Insurance claim of the Non-Life Insurance Pursuant to sub-rule (1), the Insurer shall immediately designate a Surveyor to make necessary inquiry, if necessary
	\item The Surveyor deputed pursuant to sub-rule (2) shall make necessary inquiry and shall determine the liability of the Insurer within fifteen days and shall submit a report to the Insurer including the comprehensive details and also inform the Insured relating to it mentioning the amount to be received by the Insured subject to the terms and conditions
	and facilities of the Insurance Policy
	\item The Insurer shall determine the liability and shall provide the payment against the claim of the Non-Life Insurance to the Insured generally within thirty-five days from the submission of the report by the surveyor pursuant to sub-rule (3).
\end{enumerate}
%
\section{Review Problems}